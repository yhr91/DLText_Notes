\documentclass{article}
\usepackage[utf8]{inputenc}

\usepackage{natbib}
\usepackage{graphicx}
\usepackage{fancyhdr}

\pagestyle{fancy}
\fancyhf{}
\fancyfoot[L]{Unless otherwise specififed, all content is the author's summary of "Goodfellow, Ian, Yoshua Bengio, and Aaron Courville. Deep learning. MIT press, 2016."}

\title{Generative Networks}
\author{yusuf.roohani }
\date{August 2017}

\begin{document}
\maketitle

\section{Generative Adversarial Networks}

\subsection{Learning from Simulated and Unsupervised Images using Adversarial Training \cite{shrivastava2016learning}} 

Learning from synthetic images often does not achieve the desired performenace due to a gap between synthetic and real image distributions. 

\begin{itemize}

    \item Propose simulated + unsupervised learning that uses unlabelled real data to refine synthetic images
    \item Train a refiner network to add realism to synthetic images using a combination of adversarial and self-regularization loss
    \item Modify GAN to stabilize training and reduce artifacts
    \item Show improved realism and state-of-the-art results on training deep neural networks

\end{itemize}

\section{Springenberg 2016}

\subsection{GANs}

Let $\chi = {x^1, x^2 ... x^N}$

The GAN objective could be written as:
$ \min_{G} \max_D $

\subsection{CatGANs}

Trying to learn a discriminative classifier using GAN's that calssifies inputs into multiple classes as opposed to a single class.

This is similar to a probabilitic cluster assigment but by using a discriminator trained 'adversarially', it will hopefully become more robust against overfitting to noise

The second observation is that a standard GAN cannot solve this multiclass problem - because it is only trained for a binary classification



\bibliographystyle{plain}
\bibliography{references}

\end{document}